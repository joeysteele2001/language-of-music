\documentclass[conference, 12pt]{IEEEtran}

\usepackage{cite}
\usepackage{url}
\usepackage{amsmath,amssymb,amsfonts}
\usepackage{enumitem}
\usepackage{algorithmic}
\usepackage{graphicx}
\usepackage{textcomp}
\usepackage{xcolor}
\def\BibTeX{{\rm B\kern-.05em{\sc i\kern-.025em b}\kern-.08em
    T\kern-.1667em\lower.7ex\hbox{E}\kern-.125emX}}
    
\newcommand\todo[1]{\textcolor{red}{#1}}
    
\begin{document}

\urlstyle{tt}

\title{Project Proposal Paper}

\author{
    \IEEEauthorblockN{Jessica Maebius}
    \and
    \IEEEauthorblockN{Jessica Paley}
    \and
    \IEEEauthorblockN{Joseph Steele}
}

\maketitle

\section{Introduction to Context}
Most people know that learning a new language can be extremely challenging, but most do not think about how music can actually make the process significantly easier.
Language teachers in elementary schools all the way to college professors often recommend students to listen to music in the language they want to learn to get more constant exposure to the pronunciations, idioms, and phrases used, while also triggering a student's memory through the musical repetition.
In this paper we outline a project that acts as a tool for teenage and adult language-learners to find songs in another language, get line-by-line translations of the lyrics, test their knowledge of the lyrics' vocabulary, and—for the musically inclined—play or sing along with the song chords.
All of these functionalities will be in the form of a web application with an interactive user interface organized into the three modes ``Just Listen,'' ``Listen \& Learn,'' and ``Play Along,'' each with access to a song library with the original and translated lyrics.
We will first review the motivations, applications, demonstration of need of this application, then go more in depth about our project goals, method, evaluation, resources, and deliverables within the scope of this semester long project.

\section{Motivation}
The motivation for this project is to maximize the usefulness of music as a language-learning tool. 
There are so many students of language who already employ music as a means of increasing their exposure to the language through passive listening.
Anyone can find songs in another language on their music software of choice, but to figure out what the lyrics actually mean requires the additional step of using a language dictionary: a step usually skipped since they are just listening passively. 
However, our application would combine a music library with a built-in translator, giving users the ease of access to phrase translations that would help significantly motivate them to look into the lyrics, thus improving their language comprehension.

Additionally, as students of both music and language, the idea of being able to play along with a song as a way to interact with the language musically is very appealing. 
Most musicians have at least some experience playing songs in other languages, and it is easy to see how that level of exposure and repetition as you practice would give you a better understanding of the language. 
The ``Play Along'' feature of our application was inspired by this connection, as it provides a space for musicians of any level to interact with the music, while also having easy access to the translations.


\section{Related Work and Demonstration of Need}
Upon researching similar products, we found two major applications with the same motivation to teach language through music. 
However, while the concept of learning a language with displayed song lyrics is similar to our project, their implementation was different. 
Both \url{lyricstraining.com} \cite{lyrics-training} and the Lirica application \cite{lirica} take an approach in which the song is broken down into lines, and the song will stop while users fill in the blank lyrics. 
Thus, they both provide an effective tool in language hearing and pronunciation, but they lose the benefits of musical repetition in the process.
The point of using music as a learning tool is based on the fact that music is built off of rhythmic and melodic patterns that trigger a listener's memory.
When a song is continually being paused, that pattern is lost. 
This is where the need arises for a musical language-learning platform that allows users to listen continually to their songs.
Our project would fill this gap. 
Additionally, from these similar resources, we recognized the need of having some way to test a user's lyric comprehension without having to pause the song.
We found a simple solution in our ``Listen \& Learn'' section, which allows users to have a short quiz to test on the vocabulary of a song after they have listened to it. 

\section{Applications}
The application of this project is that it can be used as a resource to assist with language-learning beyond the classroom. 
It particularly targets language students who enjoy listening to or playing music and want to use this to improve their language skills. 
A full version of the application would include a full uncensored song library and would therefore not be as suitable for children but would be more useful to high school/college students who are taking language classes or adults who are trying to learn a second language on the side. 

\section{Goals}
The goal of our project is to provide an accessible tool that assists people who love music, especially teenagers and adults, in learning real written and spoken foreign languages through song. 
We emphasize that popular music is a microcosm of real colloquial language, portraying how people actually speak their languages.

More concretely, we intend to develop a web application that contains an expandable song library. 
To constrain the scope of our project, we will primarily focus on English speakers learning Japanese. 
For each song, the tool will display scrolling lyrics with optional phonetic pronunciations. Users will have the ability to translate any line of lyrics on demand. 
We will hand-select at least four songs (preferably more) for high-quality translations and quizzes.

We also plan to give users a ``listen and learn'' mode, in which key vocabulary words and phrases are highlighted in the lyrics. 
After the song has played, users can take a short quiz to test their knowledge of these phrases. 
Each song's vocabulary quiz will have at least two levels of difficulty to accommodate for both new and intermediate learners.

Finally, users will have the option to view a song's lyrics with chord symbols in a third ``play along'' mode. 
Learners can cement their knowledge by playing and singing the song themselves.

\section{Proposed Method}
\subsection{Tasks and Timeline}
Our project is divided into five phases. At the end of the first four phases are ``checkpoints''; the final paper is due at the end of Phase 5. We intend to perform the following tasks during these phases:

\begin{enumerate}[label=Phase \arabic*., leftmargin=*]
\item Complete Project Proposal Presentation and Paper; Create Just Listen UI; Create Quiz UI; Design consistent color palette and fonts. \textit{(Due October 1.)}

\item Finalize UI and functionality for all three modes and Quiz for one song; Improve and test UI; Fix major bugs. \textit{(Due October 15.)}

\item Deliver Midterm Presentation; Integrate automatic translation API for any song; Implement human-translated lyrics for more songs. \textit{(Due November 5.)}

\item Fix major outstanding errors; Finalize the UI and functionality. \textit{(Due November 19.)}

\item Deliver Final Presentation; Write Final Paper; Fix known outstanding bugs; Deliver final product. \textit{(Due December 15.)}
\end{enumerate}

\subsection{Necessary Roles}
We plan to divide the project into the following roles. 
Each role does not necessarily correspond to one person, and each person does not necessarily follow one role.

\begin{LaTeXdescription}
\item[Curator] identifies songs to include in the main app library and to be translated
\item[Translator] locates and edits English translations of songs and synchronizes lines of lyrics with timestamps
\item[User Interface Designer] designs and mocks up the structure and style of the user interface in detail
\item[User Interface Developer] implements the UI design in code
\item[Software Developer] implements all functionality in code, including connecting data to the UI
\item[Code Reviewer] reviews others' code changes before they are merged into the main \verb+git+ branch
\end{LaTeXdescription}

\subsection{Potential Challenges}
We expect that there will be challenges throughout the course of the semester and have taken steps to limit the scope of our project accordingly. 
We anticipate that one of the more challenging aspects of the project will be growing accustomed to using React, but given that we all have previous coding experience this learning curve should not prove to be insurmountable. 
We have chosen to create a web app as it will allow us to design for all smart phones, tablets, and computers.
Consequently, we must be continually cognizant of the fact that because we are designing for a variety of devices---thus, our interface could look different depending on the device.
We are also likely to encounter difficulty when integrating APIs into our project. 
We have considered multiple APIs that we can use as a backup to the APIs we intend to use in our project in case we realize that our initial choice was insufficient for whatever reason; for example, if we could not use YouTube's API we could substitute with Spotify or Apple's API. 
To prevent inconsistencies in our UI we plan to establish a color and font scheme early on in the project.

Our target language, Japanese, poses unique challenges in that it uses a different writing system than English. 
This system consists of two phonetic alphabets and a collection of characters from Chinese. 
It should be noted that the Chinese characters present a significant challenge to Japanese learners due to the sheer number one needs to memorize (about 2000 for fluency) and the fact that one character may have multiple readings based on the word or context in which it is used. 
We plan to combat this issue by including phonetic Japanese characters (known as furigana). This will create an additional step in our project, but for the purpose of education we have deemed it necessary to include this aid.
Additionally, we have also considered including the option for the user to view the Japanese text using romanized Japanese, but as we have identified our target audience to be people who already have some knowledge of the language we would only include romanization to increase our potential user base.

\section{Proposed Evaluation}
Our product is rooted in the idea of education, so we will continually consider the impact of our project as a tool for education.
Regarding the real world implications of our project, while it is beyond the scope of this course to run an experiment to test the effectiveness of our app for the purposes of improving pronunciation as well as language learning as a whole, our project is founded on existing research on the impact of listening to music in a foreign language on the learner's pronunciation \cite{murphey-t}.
In addition to the benefits of being exposed to new vocabulary in the more easily memorable context of a song \cite{medina-s}, hearing a native speaker's pronunciation in music leads to an improvement in the learner's own pronunciation \cite{legg-r}.
It is important to us to display the lyrics in a non-roman alphabet as we have not seen this feature adopted in similar applications such as Lyrics Training and Lirica and viewing only romanized Japanese is not helpful in achieving short-term or long-term learning goals \cite{okuyama-y}.
While other applications are focused on drawing in a larger user base we chose to ensure that users with a long-term goal of learning a language will have the best possible experience. 
We suggest that upon creating a prototype that honors the existing research outlined above we can consider our project to be a success.

\section{Resources and Justifications}

\subsection{Tools}
Our group plans to use Trello to manage and track the state of tasks for the remainder of the semester. 
We will use Git as version control so that we can track and revert changes to code.
With Git, we can also use branches to separate development of distinct features.
We will use GitHub as a central server to manage our Git repository. 
GitHub may also serve as a tool for tracking bugs (with Issues) and performing code reviews (through Pull Requests).

\subsection{Libraries and Frameworks}
Our web app will fall into a typical web development framework, with a main HTML webpage being served, calling on our JavaScript code. 
We may use typical tools to optimize our workflow for development with features such as live-reload and debugging. 
These tools may also facilitate building our final production-mode web app.

The main library we intend to use is React. 
React is a JavaScript library that facilitates writing interactive and dynamic user interfaces for the Web.
In particular, as compared to plain JavaScript, React excels at maintaining apps with complex interfaces that remain properly synchronized to the data backing them.
React is a mature library with many learning resources, and it is relatively easy to learn for people with some programming experience. 
We believe that, despite the additional learning curve, React will be more effective than plain JavaScript in the long-term due to the complexity of manually replicating React's functionality.
In addition, in the future, should we assess that React is too burdensome to continue using, it is feasible that we may fall back to utilizing plain JavaScript to create the interactive components of our application.

\subsection{Languages}
We intend to use the JavaScript language for our project, in part due to the React library.
We will also incorporate JSX, an extension to JavaScript that enables writing React components in a declarative syntax similar to HTML. 
One group member has some experience developing in JavaScript. 
All members have used Python and other languages with syntaxes similar to JavaScript (such as Java and C).

Our group may also utilize Typescript, another extension to JavaScript which checks the types of variables at compile-time.
A static type checker helps detect common programming errors before they are accidentally integrated into the rest of the code and cause subtle bugs.
Typescript type annotations can be gradually integrated into a code base, so these additional features and complexities can be learned at any comfortable rate.

We will also need to use a data format such as JSON in order to represent song metadata, lyrics, translations, phonetic spellings, lyric timings, etc.
Such a format will allow for easy storage, retrieval, and editing of data.
JSON will likely be the best choice since it is closely related to JavaScript.
Since we will already be familiar with JavaScript, learning JSON will require minimal additional effort.

\section{Deliverables}
At the end of the semester, we will have produced a web app that can be used on mobile phones and computers. 
We will deliver the compiled files that can be directly opened in a web browser.
Alternatively, we may pursue hosting the app publicly on, e.g. GitHub Pages.

In addition to the app itself, the group will turn in a link to a public repository for all code and other resources used to compile the app.

\begin{thebibliography}{00}
\bibitem{legg-r} Legg, R. (2009). Using Music to Accelerate Language Learning: An Experimental Study. Research in Education, 82(1), 1--12. \url{https://doi.org/10.7227/RIE.82.1}
\bibitem{lirica} Lirica. (n.d.). Retrieved September 19, 2021, from \url{https://www.lirica.io/about}
\bibitem{lyrics-training} Lyrics Training. (n.d.). Retrieved September 19, 2021, from \url{https://lyricstraining.com/}
\bibitem{medina-s} Medina, S. L. (1990). The Effects of Music upon Second Language Vocabulary Acquisition. \url{https://eric.ed.gov/?id=ED352834}
\bibitem{murphey-t} Murphey, T. (1990). Song and Music in Language Learning. Peter Lang. \url{https://www.peterlang.com/view/title/3607}
\bibitem{okuyama-y} Okuyama, Y. (2007). CALL Vocabulary Learning in Japanese: Does Romaji Help Beginners Learn More Words? CALICO Journal, 24(2), 355--379.
\end{thebibliography}

\end{document}
